\documentclass {article}
\usepackage {geometry}
\usepackage {CJK}
\usepackage {amsmath}
\usepackage {amssymb}
\usepackage {indentfirst}
\usepackage {listings}
\usepackage {courier}
\usepackage[table,xcdraw]{xcolor}

\lstset{basicstyle=\ttfamily,breaklines=true,numbers=left}

\begin{document}
  \begin {CJK*} {UTF8} {gbsn}
    \title {\textbf {\Huge Homework 10}}
		\author {郭天魁 \\ 信息科学技术学院 \\ 1300012790}

		\maketitle

		\section{Homework 10}
			\subsection{6.23}
				容量$V=C\cdot xr\cdot (1-x)r$,其中$C$为一常数,易知$x=0.5$时取到最大值。\\

			\subsection{6.25}
				A.

				$T_{\text{max rotation}}=\dfrac{1}{\text{RPM}}\cdot \dfrac{60000\text{ms}}{1\text{min}}=4\text{ms},$

				$T_{\text{avg rotation}}=\dfrac{1}{2}T_{\text{max rotation}}=2\text{ms},$

				按照书上6.4的计算方式,2MB的文件包含4000个512-byte的逻辑块。

				在最好的情况下,当找到第一个扇区后,磁盘只需要转4圈来读取所有块即可,则读这个文件的总时间为$T_{\text{avg seek}}+T_{\text{avg rotation}}+4*T_{\text{max rotation}}=22\text{ms}.$\\

				B.

				按照书上6.4的计算方式,总时间为$4000*(T_{\text{avg seek}}+T_{\text{avg rotation}})=24000\text{ms},$
			
				$T_{\text{transfer}}=4*T_{\text{max rotation}}=16\text{ms}$因为太小而不需要计算在内。\\

			\subsection{6.27}
				\begin{table}[h]
					\begin{tabular}{ccccccccc}
						Cache & $m$ & $C$                                       & $B$                                     & $E$                                    & $S$                                      & $t$                                     & $s$                                    & $b$                                    \\ \hline
						1.    & 32  & {\color[HTML]{34CDF9} $\underline{2048}$} & 8                                       & 1                                      & {\color[HTML]{34CDF9} $\underline{256}$} & 21                                      & 8                                      & 3                                      \\
						2.    & 32  & 2048                                      & {\color[HTML]{34CDF9} $\underline{4}$}  & {\color[HTML]{34CDF9} $\underline{4}$} & 128                                      & 23                                      & 7                                      & 2                                      \\
						3.    & 32  & 1024                                      & 2                                       & 8                                      & 64                                       & {\color[HTML]{34CDF9} $\underline{25}$} & {\color[HTML]{34CDF9} $\underline{6}$} & 1                                      \\
						4.    & 32  & 1024                                      & {\color[HTML]{34CDF9} $\underline{32}$} & 2                                      & 16                                       & 23                                      & 4                                      & {\color[HTML]{34CDF9} $\underline{5}$}
					\end{tabular}
				\end{table}

			\subsection{6.29}
				A. Nothing.\\

				B. 0x18F0, 0x18F1, 0x18F2, 0x18F3, 0xB0, 0xB1, 0xB2, 0xB3.\\

				C. 0xE34, 0xE35, 0xE36, 0xE37.\\

				D. 0x1BDC, 0x1BDD, 0x1BDE, 0x1BDF.\\

			\subsection{6.46}
				\begin{lstlisting}[language=C]
void transpose_fast(int *dst, int *src, int dim) {
	int i, j, k, l, si, sj;
#define blocksizex 800
#define blocksizey 400
#define min(x,y) ((x) < (y) ? (x) : (y))
	for (i = 0; i < dim; i += blocksizex) {
		for (j = 0; j < dim; j += blocksizey) {
			si = min(i + blocksizex, dim);
			sj = min(j + blocksizey, dim) - 1;
			for (k = i; k < si; k++) {
				for (l = j; l < sj; l += 2) {
					dst[l*dim + k] = src[k*dim + l];
					dst[(l + 1)*dim + k] = src[k*dim + l + 1];
				}
				if (l == sj) {
					dst[l*dim + k] = src[k*dim + l];
				}
			}
		}
	}
}
				\end{lstlisting}
				与原函数相比,加速比稳定在2.8以上。\\

  \end {CJK*}
\end {document}

