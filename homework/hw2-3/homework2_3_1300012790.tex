\documentclass {article}
\usepackage {geometry}
\usepackage {CJK}
\usepackage {amsmath}
\usepackage {amssymb}
\usepackage {indentfirst}
\usepackage {listings}
\usepackage {courier}

\lstset{basicstyle=\sffamily,breaklines=true,numbers=left}

\begin{document}
  \begin {CJK*} {UTF8} {gbsn}
    \title {\textbf {\Huge Homework 2 \& Homework 3}}
	\author {郭天魁 \\ 信息科学技术学院 \\ 1300012790}

	\maketitle

	\section{Homework 2}
	  \subsection{2.85}
	    \begin{table}[h]
		  \begin{tabular}{lcc}
			                                 & \multicolumn{2}{c}{Extended precision}                      \\ \cline{2-3} 
			  Description                    & Value                         & Decimal                     \\ \hline
			  Smallest positive denormalized & $2^{-16445}$                  & $3.645200\times 10^{-4951}$ \\
			  Smallest positive normalized   & $2^{-16382}$                  & $3.362103\times 10^{-4932}$ \\
			  Largest normalized             & $(2-2^{-63})\times 2^{16382}$ & $5.948657\times 10^{4931}$ 
		  \end{tabular}
	    \end{table}

	  \subsection{2.93}
	    \begin{lstlisting}[language=C]
float_bits float_half(float_bits uf) {
	unsigned sign = uf & 0x80000000;
	unsigned exp = uf & 0x7F800000;
	unsigned frac = uf & 0x7FFFFF;
	unsigned round = (frac & 3) == 3;
	unsigned frac_half = (frac >> 1) + round;
	if (exp == 0x7F800000)
		return uf;
	if (exp)
		frac = (exp -= 0x800000) ? frac : 0x400000 + frac_half;
	else
		frac = frac_half;
	return sign | exp | frac;
}
        \end{lstlisting}
		在$f$为\textit{NaN}时,直接返回$f$与直接进行浮点运算的结果不符,具体规则待考。\\

	  \subsection{2.95}
	    \begin{lstlisting}[language=C]
float_bits float_i2f(int i) {
	unsigned sign = i & 0x80000000u;
	unsigned f = i;
	if (!i)
		return 0;
	if (sign)
		f = (~f) + 1;
	int exp = 0x4F000000;
	while (~f & 0x80000000) {
		f <<= 1;
		exp -= 0x800000;
	}
	f ^= 0x80000000;
	f += 0x7f;
	if ((f & 0x1ff) == 0x1ff)
		f++;
	f >>= 8;
	return sign | (exp + f);
}
	    \end{lstlisting}
		通过全部测试。\\

  \section{Homework 3}
    \subsection{3.55}
	  注意到对于64-bit的数$x$,其作为有符号形式所表示的数与无符号形式所表示的数对于$2^{64}$是同余的,所以如果$x$是64-bit的,那么只需要执行$x$与$y$的无符号乘法并使其自然溢出即可。此时令$x_1,x_2,y_1,y_2$分别表示$x,y$的高位与低位,$A_1,A_2$分别表示答案$xy$的高位与低位,则有
	  $$A_1=x_1y_2+x_2y_1,A_2=x_2y_2,$$

	  其中$A_2$可能向$A_1$溢出,$A_1$可能自然溢出。\\

	  而此时$x$为32-bit的数,则$x_2=x,x_1=x>>31=\begin{cases}
		  0, & x\geq 0,\\
		  -1, & x<0.
	  \end{cases}$\\

	  以下为算法过程:
	\begin{lstlisting}[language={[x86masm]Assembler}]
movl	16(%ebp), %esi		# get y2
movl	12(%ebp), %eax		# get x2
movl	%eax, %edx
sarl	$31, %edx		# x1 = x >> 31
movl	20(%ebp), %ecx		# get y1
imull	%eax, %ecx		# calculate x2y1
movl	%edx, %ebx
imull	%esi, %ebx		# calculate x1y2
addl	%ebx, %ecx		# A1 = x1y2 + x2y1
mull	%esi			# A2 = [edx:eax] = x2y2
leal	(%ecx,%edx), %edx	# add A1 to edx
movl	8(%ebp), %ecx		# get dest
movl	%eax, (%ecx)		# save xy to *dest
movl	%edx, 4(%ecx)
	  \end{lstlisting}

    \subsection{3.56}
	  A. \%esi - x, \%ebx - n, \%edi - result, \%edx - mask.\\

	  B. -1(0xffffffff) - result, 1 - mask.\\

	  C. mask != 0.\\

	  D. mask = mask \textless\textless\ n. 在执行过程中,实际只考虑n的后8位。\\

	  E. result \textasciicircum = x \& mask.\\

	  F.
	  \begin{lstlisting}[language=C]
int loop(int x, int n)
{
	int result = -1;
	int mask;
	for (mask = 1; mask != 0; mask = mask << n) {
		result ^= x & mask;
	}
	return result;
}
	  \end{lstlisting}

	\subsection{3.58}
	  \begin{lstlisting}[language=C]
int switch3(int *p1, int *p2, mode_t action)
{
	int result = 0;
	switch(action) {
		case MODE_A:
			result = *p1;
			*p1 = *p2;
			break;
		case MODE_B:
			result = *p1 + *p2;
			*p2 = result;
			break;
		case MODE_C:
			*p2 = 15;
			result = *p1;
			break;
		case MODE_D:
			*p2 = *p1;
			result = 17;
			break;
		case MODE_E:
			result = 17;
			break;
		default:
			result = -1;
	}
	return result;
	  \end{lstlisting}

	\subsection{3.60}
	  A. \&A[i][j][k]$=x_A+L(i\cdot S\cdot T+j\cdot  T+k)$,其中$x_A$为A的起始地址,$L$为类型长度,此题中为$4$。\\

	  B. $T=1+2\times(4+1)=11,S=\dfrac{99}{T}=9,R=\dfrac{1980}{99\times 4}=5.$\\

	\subsection{3.63}
	  \begin{lstlisting}[language=C]
#define E1(n) (2 * (n) + 1)
#define E2(n) (3 * (n))
	  \end{lstlisting}

  \end {CJK*}
\end {document}

